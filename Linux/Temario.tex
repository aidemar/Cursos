\documentclass[12pt, letter-paper]{article}
\usepackage[inner=1.5cm,outer=1.5cm,top=2.5cm,bottom=2.5cm]{geometry}
\pagestyle{empty}
\usepackage{graphicx}
\usepackage{fancyhdr, lastpage, bbding, pmboxdraw}
\usepackage[usenames,dvipsnames]{color}
\usepackage[utf8]{inputenc}
\definecolor{darkblue}{rgb}{0,0,.6}
\definecolor{darkred}{rgb}{.7,0,0}
\definecolor{darkgreen}{rgb}{0,.6,0}
\definecolor{red}{rgb}{.98,0,0}
\usepackage[colorlinks,pagebackref,pdfusetitle,urlcolor=darkblue,citecolor=darkblue,linkcolor=darkred,bookmarksnumbered,plainpages=false]{hyperref}
\renewcommand{\thefootnote}{\fnsymbol{footnote}}

\pagestyle{fancyplain}
\fancyhf{}
\lhead{ \fancyplain{}{Linux} }
\rhead{ \fancyplain{}{Marzo 2018} }
\fancyfoot[RO, LE] {page \thepage\ of \pageref{LastPage} }
\thispagestyle{plain}

\usepackage{listings}
\usepackage{caption}
\DeclareCaptionFont{white}{\color{white}}
\DeclareCaptionFormat{listing}{\colorbox{gray}{\parbox{\textwidth}{#1#2#3}}}
\captionsetup[lstlisting]{format=listing,labelfont=white,textfont=white}
\usepackage{verbatim}
\usepackage{fancyvrb}
\usepackage{acronym}
\usepackage{amsthm}
\renewcommand{\familydefault}{\sfdefault}

\definecolor{OliveGreen}{cmyk}{0.64,0,0.95,0.40}
\definecolor{CadetBlue}{cmyk}{0.62,0.57,0.23,0}
\definecolor{lightlightgray}{gray}{0.93}
\VerbatimFootnotes

\lstset{
  % language=bash,                        % Code langugage
  basicstyle=\ttfamily,                   % Code font, Examples: \footnotesize, \ttfamily
  keywordstyle=\color{OliveGreen},        % Keywords font ('*' = uppercase)
  commentstyle=\color{gray},              % Comments font
  numbers=left,                           % Line nums position
  numberstyle=\tiny,                      % Line-numbers fonts
  stepnumber=1,                           % Step between two line-numbers
  numbersep=5pt,                          % How far are line-numbers from code
  backgroundcolor=\color{lightlightgray}, % Choose background color
  frame=none,                             % A frame around the code
  tabsize=2,                              % Default tab size
  captionpos=t,                           % Caption-position = bottom
  breaklines=true,                        % Automatic line breaking?
  breakatwhitespace=false,                % Automatic breaks only at whitespace?
  showspaces=false,                       % Dont make spaces visible
  showtabs=false,                         % Dont make tabls visible
  columns=flexible,                       % Column format
  morekeywords={__global__, __device__},  % CUDA specific keywords
}

\begin{document}
\begin{center}
  {\Large \textsc{Linux}}
\end{center}
\begin{center}
  Para web developers
\end{center}

\begin{center}
  \rule{6in}{0.4pt}
  \begin{minipage}[t]{.75\textwidth}
    \begin{tabular}{llcccll}
      \textbf{Instructora:} & Andrea Gómez & & & &
                                                   \textbf{Fecha:} &  Marzo, 2018
    \end{tabular}
  \end{minipage}
  \rule{6in}{0.4pt}
\end{center}
\vspace{.5cm}
\setlength{\unitlength}{1in}
\renewcommand{\arraystretch}{2}

% \noindent
\textbf{Temario:}
\begin{enumerate}
\item Introducción a Linux
  \begin{enumerate}
  \item ¿Qué es un sistema operativo?
    \begin{enumerate}
    \item Windows
    \item POSIX
    \end{enumerate}
  \item Linux \& UNIX
    \begin{enumerate}
    \item ¿Qué es UNIX?
    \item ¿Qué es Linux?
    \item Diferencias entre Linux y UNIX
    \end{enumerate}
  \item Usando Linux
    \begin{enumerate}
    \item ¿Por qué usar Linux?
    \item ¿Quién usa Linux?
    \end{enumerate}
  \end{enumerate}
  
\item Distribuciones de Linux
  \begin{enumerate}
  \item ¿Qué es una distribución de Linux?
    \begin{enumerate}
    \item Similitudes entre distribuciones de Linux
    \item Diferencias entre distribuciones de Linux
    \end{enumerate}
  \item Escogiendo una distribución de Linux
    \begin{enumerate}
    \item Distribuciones más usadas para uso personal
      \begin{enumerate}
      \item Debian
      \item Ubuntu, Linux Mint, elementaryOS
      \item Fedora
      \item Arch Linux, Gentoo
      \end{enumerate}
    \item Distribuciones más usadas en servidores
      \begin{enumerate}
      \item Debian, Ubuntu
      \item RHEL
      \item CentOS, Fedora
      \end{enumerate}
    \end{enumerate}
  \end{enumerate}
  
\item Instalando Linux
  \begin{enumerate}
  \item Planeando la instalación
    \begin{enumerate}
    \item Respaldando información importante
    \item Instalando más de un sistema operativo
    \item Particionar disco duro
    \item Conociendo la configuración del equipo
      \begin{enumerate}
      \item Requisitos de hardware
      \item Firmwares UEFI y BIOS
      \end{enumerate}
    \end{enumerate}
  \item Preparando un medio de instalación
    \begin{enumerate}
    \item Descargando Linux
    \item Creando un medio de instalación
    \end{enumerate}
  \item El proceso de instalación
    \begin{enumerate}
    \item Configurando la hora y región geográfica del sistema
    \item Definiendo idioma y configuración de teclado
    \item Creando particiones de disco
      \begin{enumerate}
      \item Tablas de particiones GPT y MBR
      \item Sistemas de archivos ext4, NTFS y HFS
      \item Partición compartida con otro sistema operativo
      \end{enumerate}
    \item Usuarios
      \begin{enumerate}
      \item Creando usuarios
      \item Recomendaciones para hacer contraseñas seguras
      \item Usuario Root
      \end{enumerate}
    \item Instalando el gestor de arranque GRUB
      \begin{enumerate}
      \item Creando entradas para otros sistemas operativos en GRUB
      \end{enumerate}
    \end{enumerate}
  \end{enumerate}
  
\item Iniciando el sistema por 1era vez
  \begin{enumerate}
  \item El gestor de arranque
  \item El ``entorno'' Linux
  \item Iniciando sesión con nuestro usuario
    \begin{enumerate}
    \item La carpeta ``home''
    \end{enumerate}
  \item Actualizando el sistema
  \end{enumerate}
  
\item La terminal
  \begin{enumerate}
  \item ¿Qué es la terminal y para qué sirve?
  \item Iniciando la terminal
  \end{enumerate}
  
\item La Shell
  \begin{enumerate}
  \item La Shell y su utilidad
  \item Las Shell más comunes
    \begin{enumerate}
    \item Sh
    \item Bash
    \item Zsh
    \end{enumerate}
  \item Conociendo la simbología de la Shell
  \end{enumerate}
\item Trabajando con la Shell
  \begin{enumerate}
  \item Archivos y directorios
    \begin{enumerate}
    \item Listando archivos y carpetas
    \item Cambiando de directorios
    \item Creando archivos y carpetas
    \item Copiando archivos y carpetas
    \item Moviendo archivos y carpetas
    \item Eliminando archivos y carpetas
    \item Concatenando archivos y cadenas
    \item Imprimiendo a un archivo desde la terminal
    \end{enumerate}
  \item Editores de texto más comunes en Linux
    \begin{enumerate}
    \item Nano
    \item Vim
    \item Emacs
    \end{enumerate}
  \item Usuarios y grupos
    \begin{enumerate}
    \item Usuarios
      \begin{enumerate}
      \item Creando y elimiando usuarios
      \item Permisos de usuarios
      \end{enumerate}
    \item Grupos y su utilidad
      \begin{enumerate}
      \item Creando un grupo
      \item Añadiendo y eliminando usuarios de un grupo
      \item Eliminando un grupo
      \end{enumerate}
    \end{enumerate}
  \end{enumerate}
\item Shell Scripting
  \begin{enumerate}
  \item ¿Qué es Shell Scripting?
  \item Mi primer script
    \begin{enumerate}
    \item Requisitos y elementos básicos de un script
    \item Permisos necesarios para ejecutar scripts
    \item ``Hello World''
    \end{enumerate}
  \item Redirecciones y tuberías
    \begin{enumerate}
    \item Redirección
      \begin{enumerate}
      \item Sobreescribiendo un archivo
      \item Concatenando a un archivo
      \item Redirigiendo de un archivo
      \item Redirigiendo mensajes de error
      \end{enumerate}
    \item Tuberías
    \end{enumerate}
  \item Variables, ciclos y funciones
    \begin{enumerate}
    \item Variables
      \begin{enumerate}
      \item Definiendo variables
      \item Trabajando con variables
      \item Ámbito de una variable
      \item Variables de entorno           
      \item Exportando variables
      \end{enumerate}
    \item Ciclos
      \begin{enumerate}
      \item For
      \item While
      \end{enumerate}
    \item Sentencias condicionales
      \begin{enumerate}
      \item If, then, else
      \item Switch case
      \end{enumerate}
    \item Funciones
      \begin{enumerate}
      \item Retorno de funciones
      \item Declarando funciones
      \item Invocando funciones
      \end{enumerate}
    \end{enumerate}
  \end{enumerate}
\end{enumerate}

\vskip.15in

\textbf{Descargando lo necesario:} Recomiendo llevar descargado la imagen ISO de Debian
(distribución que usaremos durante el curso) para agilizar el proceso:

\href{https://cdimage.debian.org/debian-cd/current/amd64/iso-dvd/debian-9.4.0-amd64-DVD-1.iso}{Imagen
  para 64 bits}

\href{https://cdimage.debian.org/debian-cd/current/i386/iso-dvd/debian-9.4.0-i386-DVD-1.iso}{Imagen
para 32 bits}

Es posible instalar Linux en una máquina virtual sin hacer cambios directos a
nuestras particiones usando Virtual Box, lo recomiendo si no desean instalar
Linux en su sistema pero quieren poder usarlo desde su sistema operativo
principal (ej. Windows o Mac OS):

\href{https://www.virtualbox.org/wiki/Downloads}{Instrucciones para descargar
  Virtual Box}

Si desean instalar Linux en su computadora y no usar una máquina virtual es muy
importante que respalden todos sus archivos en un lugar seguro como un disco
duro externo y que lleven una memoria USB de más de 6 Gb.

Es recomendable también que desfragmenten sus discos duros previo al curso para
poder crear particiones con mayor libertad.
\end{document} 
